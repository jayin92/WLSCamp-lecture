\chapter{碎形\\跨越維度的奇幻之旅}
\chapterauthor{陸柏丞}

\section{先備知識}
\subsection{相似}
\noindent
\subsubsection{定義:}
\vspace{2.5em}
\subsubsection{相似圖形邊長,面積,和體積的關係:}
\vspace{2.5em}
\textbf{假設圖形$F$和圖形$G$相似,$F$和$G$的相似比為$1$比$k$,那麼$F$和$G$的面積比為$1$比$k^2$;體積比為1比$k^3$。}

\subsection{指數函數}
\noindent
\subsubsection{定義:}
\subsubsection{例如:}
\subsection{無限}
\noindent
定義:
\newpage
\section{自我相似}
\subsection{製作考區曲線 (Koch Curve)}
\vspace{7.5cm}
\begin{enumerate}
\item 每一張圖的圖形總長度有規律嗎?規律是甚麼?可否解釋這個規律的來源?
\item 圖形中,那些部分有相似呢?
\end{enumerate}
\subsection{製作史賓斯基地毯 (Sierpinski Carpet)}
\vfill
\begin{enumerate}
\item 每一張圖的圖形剩餘面積有規律嗎?規律是甚麼?
\item 圖形中,那些部分有相似呢?
\end{enumerate}
\section{碎形(fractal)}
數學家:曼德布洛特 (Benoit B. Mandelbrot, 1924-2010)
定義;「一個粗糙或零碎的幾何形狀,可以分成數個部分,且每一部分都(至少近似地)是整體縮小後的形狀」考區曲線及史賓斯基地毯極為碎形的典型例子。

問:真正的碎形存在嗎?\\
其實,考區曲線和史賓斯基地毯確實存在,只是我們無法看見,活動一和活動二圖案只是重複幾何操作有限次就停止的仿冒圖。

雖然完美的碎形在生活中不存在,生活中處處皆充滿自我相似的物品,例如:花椰菜、血管分支、積雨雲、閃電、布朗運動等。而碎形理論可以完整的解釋生活中這些自我相似的事物。
\section{碎形的維度}
\subsection{問題與討論}
\begin{enumerate}
\item 考區曲線有多長呢?
\item 考區曲線包圍的面積有多大呢?
\item 史賓斯坦地毯周長有多長呢?
\item 史賓斯坦地毯面積多大呢?
\end{enumerate}
看來考區曲線和史賓斯基地毯的真正面貌與大家的印象截然不同。

\begin{itemize}
\item 考區曲線:周長無限大的鋸齒狀圖形,且貼在平面特定區域內。
\item 史賓斯坦地毯:周長無限大,面積卻是零的國王的地毯。
\end{itemize}

模糊的文字很難說明碎形真正的樣貌;然而,聰明的數學家豪斯多夫巧妙的使用維度來定義碎形。
\subsection{計算維度}
\subsubsection{維度$k$和拼排個數的法則}
把$n \cdot k$個縮小$n$倍的相似圖形加起來等於1。為了瞭解上述定義,我們來討論縮放倍率與質量的關係。
\begin{enumerate}
\item 正方形 \\

\item 正方體 \\

\item 考區曲線 \\

\item 史賓斯基地毯 \\

\end{enumerate}

\section{講師介紹}
\begin{itemize}
\item 姓名:陸柏丞
\item 性別:男(不過沒辦法定論是不是人類)
\item 特色:啊$\sim\sim\sim\sim$、對朋友的定義很奇怪,可能對他來說還沒記住名字的是"朋友"、開冷氣一定要投票、對英文雜誌情有獨鍾、喜歡跟別人在浴室聊天、會撿肥皂抹你的臉、曾提出"陸三點":要讀書、要認真、要投票
\item 名言:我今天早上只跑了五圈操場,然後去跟學長打球,我好懶惰喔。
\end{itemize}
