\chapter{我的改變,你看的見!\\基礎初等微積分}
\chapterauthor{吳冠廷}

\section{函數的極限}
只要$x$越來越靠近,$f(x)$跟極限的距離能比任何正數都還小 \\ \\
定義:\\ \\ 
表示方法:\\ \\
筆記/計算:\\
\newpage
\section{微分}
將兩個極小的數相除,形成斜率。 \\ \\
定義:\\ \\ 
表示方法:\\ \\
筆記/計算:\\
\newpage
\section{積分}
將很多極小的長條相加,形成面積 \\ \\
定義:\\ \\ 
表示方法:\\ \\
筆記/計算:\\
\newpage
\section{附錄}
\subsection{補充資料}
這邊有點難度,不過如果你是天資聰穎(桃園市科學奧林匹亞有得獎就可以了)的國中生,可以去研究一下 \\
\begin{enumerate}
\item \textbf{極限:} \\
公式、羅畢達法則、p級數的次方數與收斂發散關西、等比數列發散收斂條件、交錯級數判別法、各種審斂法
\item \textbf{微分:} \\
泰勒展開式、偏微分、隱含數微分、微分方程、柯西均值定理
\item \textbf{積分:}\\
分布積分法、變數變換法、三角積分法、積球的表面積和體積、積轉動慣量、瑕積分
\end{enumerate}
\subsection{發展學科}
有興趣可以研究、但許多都很難,所以有興趣的話可以認識一下,等數學好再回來(?) \\
\begin{enumerate}
\item \textbf{物理:} \\
力學、電磁學、熱力學、統計力學、相對論、量子力學……(簡單來說就是所有的物理領域)\url{http://bit.ly/2Ujms7f}
\item \textbf{數學:} \\
高等微積分、統計、變分法……
\item \textbf{社會科學:} \\
\url{http://bit.ly/2DcwSjI}
\item \textbf{化學:} \\
\url{http://t.cn/EXL3aB7}, \url{http://bit.ly/2UcTxly}
\item \textbf{生物:} \\
洛特卡-沃爾泰拉方程:\url{http://t.cn/EX2vYfi}\\
計算生物學:\url{http://t.cn/EX2ZDh}, \url{http://bit.ly/2Ug9aIH}
\item \textbf{工科(電機、電子、機械、土木、化工、材料…):} \\
工程數學:\url{http://t.cn/EXLm6Cc}, \url{http://t.cn/EXLu4oz}
\item \textbf{社會科學:} \\
\url{http://t.cn/EKdmEYI}
\end{enumerate}
\subsection{推薦YouTube頻道}
\begin{enumerate}
\item \textbf{3b1b:} \\
有關於微積分的全系列影片,以較視覺化的方法幫助你理解,其他的影片也都很不錯,值得一看~~
\item \textbf{中華科技大學開放式課程──微積分系列:} \\
很清楚且淺顯易懂,適合初學者
\item \textbf{blackpenredpen:} \\
有許多較難的微積分題目及講解,但為英文影片,需有英文基礎
\item \textbf{清華大學開放式課程──高淑蓉教授微積分:} \\
因為是正規課程,所以較嚴謹且有條理,缺點是影片長度長,無法快速上手 \\ \\
\end{enumerate}
\begin{center}
\textbf{\large 有電子書的需求請跟講師說,我可以分享給你~~~}
\end{center}

\section{講師介紹}
\begin{itemize}
\item 姓名:吳冠廷
\item 性別:男
\item 特色:桌面十分雜亂,寫出的字也非常瀟灑(編按:但國中時期的字非常漂亮)、具有領導力、笑聲十分經典、會在黑特武陵留言裝弱,比熊熊還誇張。
\item 名言:哈哈哈哈哈哈!!!、沒差拉那個不是重點!
\end{itemize}