\chapter{世界和平遊戲\\武陵動盪}

\section{遊戲簡介}
本遊戲為回合制遊戲,共三回合,參與小隊將代表不同國家進行遊戲,透過武陵大陸的世界觀,分隊體驗各國狀況,藉由模擬各國互動,準備到時候的緊急狀況。

每個國家會有初始國家人民與資源,包含居民、軍隊、古硬幣、糧食、及科技產品。在遊戲中,你必須盡力爭取資源,已成為進步幅度最大的國家。

國家必須解決遊戲中的危機,以利國家順利運作。你可以透過與其他國家協商獲得危機真相,以避免資源不斷損失。

另外,國家與國家間可互相結盟,亦可以對其他國家發對戰爭。贏家可以掠奪輸家的龐大資源,成為更強盛的國家。

遊戲結束時,各國將依據人民積資源的變化量獲得積分,積分最高者即為世界和平遊戲的贏家!

\section{遊戲規則}
\subsection{國家分配}
參加者將被分為七組,每個小隊會被隨機分配到一個國家,七個國家分別是:

\begin{enumerate}
    \item \textbf{育崇科學合眾國} \\
    前身為術諮班的新派勢力。\textbf{科學班}為此國的重要支柱,強調「重理論輕應用」,講求理論的嚴謹推導,建立許多公理系統。雖然武力不是非常強大,但它們\textbf{國家資產極高},透過極度嚴謹公理約束社會,使用嚴密的策略來與各國外交或對抗。
    \item \textbf{大志青暨原生消費福利聯合王國} \\
    前身為術諮班的舊派勢力。\textbf{術諮班}為此國的重要支柱,強調「重應用輕理論」,研究講求實際應用,努力在原本的基礎上追求革新,任何無實際用途的發明或研究都不被重視,也因此擁有強大的工業實力。另外,此國為原生消費福利自治區的建立國,\textbf{控制武陵大陸的糧食命脈}。
    \item \textbf{建北黑魔法帝國} \\
    此國位於荒涼的台北大莽原,創立者為\underline{\hspace{2em}},他利用神秘的黑魔法及莽原土著的負面情緒,建立了強大的建北黑魔法帝國。因為對武陵興盛的科學發展感到恐懼,此國試圖利用魔力影響武陵大陸人民的作息,以保持自身的地位及安全。而強盛的\textbf{土著勢力}及\textbf{黑科技}讓帝國齊身武力強國。
    \item \textbf{大明道普通共和國}\\
    前身為明道王國。\textbf{普通班}為此國的主要成員,普通表示普遍通用,不跟隨著科學革命的潮流,亦不固守於老舊的術諮主義,它們廣泛涉及多種知識,無論是術諮、科學、美育,各種國家的文化在此處兼容並蓄,可稱為武陵的文化大熔爐,但各知識的能力皆不專精,能力是最平均的一個國家。此國的特色為居民眾多,生產力極高。
    \item \textbf{科教聯邦}\\
    前身為三個自治州:物州、化州、生州。\textbf{設備組}為此聯邦的管理者,他們力量強大,積極保護聯邦內\textbf{豐富的自然資源},如實驗室地形,天然的實驗器材礦產,以及自然生成的資優、創發洞穴,避免珍貴資源被虎視眈眈的強國掠奪。由於他是中立國,所以是最安全的地區,是各國協商及科學推廣隊訓練的最佳選擇。 
    \item \textbf{美育聯合大公國}\\
    \textbf{音樂班}為此國主要成員。此國因為經歷過文藝復興,不崇尚科學,四處充滿著藝術風,比起武力,更注重社會上人與人之間的交際與無私的純樸。另外,優良的家庭背景使此國擁有\textbf{巨額資金}。
    \item \textbf{間諜組織}\\
    此國成員常穿梭於國於國之間搜尋情報,並以\textbf{販賣情報}維生。特殊能力:此國成員在協商階段時,如果申明自己是間諜組織的成員,就有權力詢問任何國家的成員任何問題,被詢問的成員一定要誠實回答每個問題。\\
    遊戲中,小隊將代表自己的國家,進行各項任務與動作。
\end{enumerate}

\subsection{角色分配}
\noindent
各國須選出以下官職:\\
\textbf{皇帝:}國家實質代表,負責統籌意見及下達命令。\\
\textbf{宰相:}管理人民,決定並回報國家生產物資。\\
\textbf{大都督:}管理軍隊,決定戰爭相關事物。當本國國家代表在宣言階段向其他國家開戰時,大都督有否決權。\\
\textbf{官員集團:}其餘玩家都是普通官員,可參與內政討論,負責國際談判。
\subsection{人民與資源說明}
\noindent
\textbf{居民}:國家的主要命脈,可以在每年年初生產物資。\\
\textbf{軍隊}:國家雇用的國際軍隊,負責出征國際戰爭。\\
\textbf{古硬幣}:國家間的流通資金,為國際交易的媒介。\\
\textbf{糧食}:居民及軍隊的維生必需品。\\
\textbf{科技產品}:國家的武力象徵,為發動戰爭的消耗品。\\

下表是各個國家的初始人民及資源:
\begin{table}[H]
\centering
\resizebox{\textwidth}{!}{%
\begin{tabular}{|c|c|c|c|c|c|c|c|}
\hline
 & \begin{tabular}[c]{@{}c@{}}育崇\\ 科學合眾國\end{tabular} & \begin{tabular}[c]{@{}c@{}}大志青暨\\ 原生消費\\ 福利聯合王國\end{tabular} & \begin{tabular}[c]{@{}c@{}}建北\\ 黑魔法帝國\end{tabular} & \begin{tabular}[c]{@{}c@{}}大明道\\ 普通共和國\end{tabular} & \begin{tabular}[c]{@{}c@{}}科教\\ 聯邦\end{tabular} & \begin{tabular}[c]{@{}c@{}}美育\\ 聯合大公國\end{tabular} & \begin{tabular}[c]{@{}c@{}}間諜\\ 組織\end{tabular} \\ \hline
居民(萬位) & 20 & 18 & 22 & 32 & 12 & 14 & 12 \\ \hline
軍隊(萬位) & 10 & 9 & 22 & 16 & 6 & 7 & 12 \\ \hline
古硬幣(萬元) & 240 & 72 & 88 & 128 & 48 & 168 & 48 \\ \hline
糧食(萬) & 160 & 432 & 176 & 256 & 96 & 112 & 96 \\ \hline
科技產品(單位) & 240 & 216 & 264 & 384 & 432 & 168 & 144 \\ \hline
\end{tabular}%
}
\end{table}

\subsection{回合流程}
\noindent
以下流程稱為一回合
\begin{table}[H]
    \centering
    \resizebox{\textwidth}{!}{
    \begin{tabular}{|c|c|c|c|}
    \hline
    \textbf{階段名稱} & 說明 & 時間 & 備註 \\ \hline
    \textbf{生產階段} & \begin{tabular}[c]{@{}c@{}}軍隊徵撫\\   居民生育\\   資源生產\end{tabular} & 1min 各國討論 & 各國宰相起立,向小天使回報數據。 \\ \hline
    \textbf{內政階段} & 各國內部討論方針 & 1 min(第一回合有3 min) & 國與國間禁止交談互動 \\ \hline
    \textbf{協商階段} & 各國自由協商,可以討論危機、結盟等議題 & 5 min &  \\ \hline
    \textbf{宣言階段} & \begin{tabular}[c]{@{}c@{}}危機處理\\   資源贈送\\   國家結盟\\   發動戰爭\end{tabular} & 1 min/國家 & 讓渡制度 \\ \hline
    \textbf{戰爭階段} & \begin{tabular}[c]{@{}c@{}}投入軍備\\   宣布戰爭結果\end{tabular} &  & 沒有戰爭即跳過 \\ \hline
    \textbf{結算階段} & \begin{tabular}[c]{@{}c@{}}審查各個危機狀態\\   清算各國財產(包含資源的生產及消耗)\end{tabular} &  & 小天使負責 \\ \hline
    \end{tabular}
    }
\end{table}

\subsubsection{生產階段}
各國可在此階段進行以下動作:
\begin{enumerate}
    \item \textbf{軍隊徵撫:}各國可以將人民編入軍隊,或將軍隊變為居民,每調動1萬軍隊需花費4萬古硬幣。
    \item \textbf{居民生育:}居民可以生育國家心血,各國居民增加30\%。
    \item \textbf{資源生產:}居民另可生產古硬幣、糧食、科技產品(擇一),每一萬人可生產3萬古硬幣、6萬糧食、或9單位科技產品,並馬上獲得所有資源。(國家可以分配資源生產比例)(18)
\end{enumerate}

待一分鐘討埨時間結束後,各國宰相起立,向小天使回報軍隊徵撫數量及生產的資源。

\subsubsection{內政階段}
此階段中,各國內部可以討論危機內容及國際情勢,並決定下一個協商階段要做的行動,然而,國與國之間禁止發生任何形式的交談。內政階段第一回合為三分鐘,接下來的回合皆為一分鐘。

\subsubsection{協商階段}
此階段是國與國的交流時間。各國官員可拜訪他國,一同討論國家危機、戰爭、或結盟。此階段的時間為五分鐘。

\subsubsection{宣言階段}
每一個國家可選擇一個代表,在國際舞台發聲,時間為\textbf{一分鐘},這位代表可宣布以下行動:\\

\begin{enumerate}
    \item \textbf{危機處理:}我國危機是否已解除?我國是否知道他國危機的真相?(如果危機內容必須調查出隱藏的真相,則國家代表一回合只能回答一次危機真相。)
    \item \textbf{贈送資源:}將一定數量的居民、軍隊、古硬幣、糧食、科技產品無條件贈送給別國。
    \item \textbf{國家結盟:}國家可以邀請其他國家一同結盟,如果在台下的國家答應,則兩國順利結盟。
    \item \textbf{發動戰爭:}對其他國家或國家聯盟宣戰!
\end{enumerate}

如果代表有剩餘時間,可選擇將剩餘時間轉讓給小天使,或轉讓給指定國家,被指定的國家可選擇利用剩餘的時間上台發表,或是將剩餘時間轉讓給小天使。\\

各國可選擇發聲,亦可不發聲。待各國代表宣布完畢,此階段即結束。

\subsubsection{戰爭階段}
如果協商階段中有國家對其他國家發對戰爭,則協商階段結束後將進入戰爭階段;如果沒有,則直接跳至結算階段。

戰爭雙方可決定派遣多少兵力投入戰爭,小隊必須派大都督至小天使處派遣兵力,每1萬軍隊需消耗10單位科技產品。

戰爭中的贏家為電腦隨機產生,國家戰爭勝利機率與派遣兵力成正比;另外,任務的特殊加成也可以使戰勝機率提高。

戰爭結束後,贏家居民減少10\%,失去投入軍隊20\%,輸家居民減少15\%,失去投入軍隊30\%,另外,贏家可獲得輸家25\%的古硬幣、糧食、及科技產品。(輸家聯盟每一國須損失25\%的財產,贏家聯盟則均分獲利)

當輸家的資源數量小於0,贏家無法從輸家掠奪資源。

\subsubsection{結算階段}

此階段中,小天使將進行以下動作:
\begin{enumerate}
    \item \textbf{危機檢查:}各國危機如果順利解除,則頒發相對應獎勵;如果仍未解除,則扣除相對應資源。
    \item \textbf{吃飯時間:}居民需消耗糧食,每萬居民消耗2萬糧食。軍隊須常態性消耗古硬幣和糧食,每萬軍隊消耗4萬糧食。如果糧食或金錢不足,部分飢餓的居民及軍隊將在此階段死亡。(-6), (-12)
    \item \textbf{滅國制度:}若一個國家中沒有任何居民及軍隊人數,則此國家宣布滅國,在接下來的回合中不參與遊戲。
    此階段結束後,本回合結束,進入下一個回合。
\end{enumerate}

\subsection{積分計算}

遊戲過程中,每個國家會有國家積分,公式如下:
\begin{center}
國家積分=居民變化比例+軍隊變化比例+古硬幣變化比例+糧食變化比例+科技產品變化比例 
\end{center}

其中,變化比例指的是當前資源數量與初始資源數量的差值。

當遊戲第三回合結束時,國家將會被依據遊戲積分高低而排名,最高分的國家即為世界和平遊戲中的霸主!

\section{國家危機}
\subsection{危機說明}
危機是一個國家遇到的問題,它的嚴重性足以負面的影響國家的人民及資源損失;因此,遊戲中的國家應努力解決危機,以穩定國家人民及資源的穩定性。

遊戲中總共有七個危機,每個國家有自己的主要危機,也同時干涉到多個不同其他國家的危機。國家解決危機的方式有很多種,例如:搜尋真相、國家結盟、甚至發動戰爭。

有些國家握有非常有價值的危機真相,這些珍貴情報可以被用來當作國與國間交易的物品,換取大量的資源或利益。

\subsection{危機格式}
\begin{itemize}
    \item 背景
    \item 現況
    \item 解決辦法
    \item 危機解除效果
    \item (真相)
\end{itemize}

