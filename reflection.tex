\chapter{感言}
\section{總召組}
\begin{itemize}
\item \textbf{智者(班導)} \\
武陵高中自創辦科學班,至這屆新生已是第九屆,這麼多年過去了,我們也看到了科學班的學生,在國家的資源下,一個個希望蔚然成林。

在為期四天的科學營活動中,第八屆科學班同學在營隊的籌畫與舉辦中,從追求小我的成功擴及成就社會的公義;並能充分的掌握知識、分析知識、應用知識,設計有趣的實作課程,展現創造力,讓科學概念向下紮根。

誠如王政忠老師所說:「陽光、空氣和水從不會挑選種子,教育或者教育人員也應該是。這一片廣闊的土地,在各界的灌溉下,的確開出了一簇簇迎風招展的向陽花朵,教育是陽光、是空氣、是水,只要公平的對待,每一顆或胖或瘦、或美或醜的種子,都會開花,或高或矮,或紅或白,都會開花。」

\item \textbf{炭炭(總召)} \\
雖然是總召,但是說真的,這次的營隊我並不是最大功臣。很感謝副召跟智閎努力的規劃了整個營隊的行程跟進度,也謝謝大家都非常用心的策劃這整個活動 各司其職。

把一個這麼大型的活動從無到有弄出來真的很不容易。熬了好幾個星期的夜為了設計講義;蛋糕課程從食譜設計、食材揀選到工具採買全部一手包辦,扛著好幾袋的工具跟食材跑遍了整個中壢,還有好多好多工作  主持、練舞、美宣......
雖然聽起來已經做了不少,但真的有人比我做了更多 冠廷、智閎、柏偉、安磊、杰穎等等。

辦好一個營隊遠比我想像中再難很多,但203的大家真的都很優秀
營隊的每樣規劃每堂課程都讓我大開眼界,很開心可以跟這麼多優秀的人一起合力完成這個活動。

一年前 我們在一場考試上相遇
各自的時間長河 匯入了一道我們將攜手共闖的海底隧道
挺過了最初的疏離 用合作 激盪出熠熠的火花 劃破深海的幽暗死寂
加油打氣 安慰擁抱 我們的感情也隨著原本冰冷的隧道一同升溫
走過了無數的爭吵 以原諒 縫補最難得的友誼 拼成腦海中的記憶色塊
切磋砥礪 左挈右提 將我們的青春畫布漾成色彩斑斕的海底世界
願這次的營隊 能為我們的高中生涯再添一筆輝煌的紀錄
也盼往後的兩年 我們能並肩譜出更多的回憶樂章

\item \textbf{罐頭(副召)}
歡迎也謝謝你們來,祝你們在這裡看到不同的人事物,也學會看到和了解自己。
\end{itemize}

\section{活動組}
\begin{itemize}
\item \textbf{兔兔}\\
幾個月的焦頭爛額還有最後的趕工、一群科班人生出了這個有精緻細節也有大洞的營隊。一群人可以走的很遠 很遠 而我們班盡力的把天窗補完了(?)。能跟班上一起舉辦科學推廣隊,是一種暢快!\\
By 本來想在感言放笑話但截稿想不到笑話的兔兔

\item \textbf{矮子} \\
請看 \autoref{justin-reflection}。

\item \textbf{啊$\sim\sim\sim\sim\sim$} \\
科學推廣隊四天是第八屆科學班半年努力的結晶。我們徒手設計課程,構想背景故事及相關遊戲,也製作精美的講義及美工物品。在一次次會議中,我們將爭吵化為共識,將言語轉為行動,終於完成了這個史詩級的科學營隊。希望參加的所有國中生可以在這裡享受科學的美好,並從中獲得啟發!

\end{itemize}

\section{美宣組}
\begin{itemize}
\item \textbf{圖書研究社} \\
謝謝各位來參加這次科推,相信你們能學到很多東西,也祝你們玩得愉快。
\item \textbf{阿嬤}\\
我還是覺得我很正
\item \textbf{板哥}\\
你好我好大家好 我是美宣組的(冗員)李訓至~ 關於這次『科學推廣隊』的準備的話呢~我maybe、也許、有可能;疑似、似乎、並沒有;做太多的事情呢(廢)。我認為呢,本次活動最大的宗旨,就是能讓各位學員們能在歡樂的氣氛、輕鬆又不失嚴謹的態度去進行那些,我們(他們)精心準備的各項活動(折磨)及各種精深的科學知識課堂(睡覺Zzz)的學習喔!  總而言之言而總之的話呢~我非常感謝各位能在這炎熱的暑假期間,能來到我們科學推廣隊!盡情的享受電神們的教導(折磨)吧$\sim\sim$
\end{itemize}

\section{庶務組}
\begin{itemize}
\item \textbf{阿笠博士} \\
有人說:羅馬不是一天造成的\\
我們說:科推不是一天做好的

自三月以來,科推開始慢慢地 \\
萌芽\hspace{1em}發展\hspace{1em}成形 \\
現已到暑期,距離科學推廣營 \\
只剩\hspace{1em}不到\hspace{1em}三周 

安磊編寫的精彩劇情 \\
冠廷用心的縝密規劃 \\
那一滴滴心血 \\
造就偉大目標 

如今\hspace{1em}大家開始啟動 \\
那份來自內心的衝勁 \\
所以\hspace{1em}我們盡心盡力 \\
一起貢獻隱藏的力量

物理\hspace{2em}數學 \\
\phantom{活動}生輔 \\
生物\hspace{2em}化學

組成宇宙的\hspace{1em}平衡常數

活動\hspace{2em}庶務 \\
\phantom{活動}等等 \\
攝影\hspace{2em}隊輔

建造神聖的\hspace{1em}武陵帝國


願\hspace{1em}大家 \\
努力到最後 \\
科學推廣隊 \\
終\hspace{1em}完美

\item \textbf{比例} \\
科推除了課程外還有很多很多的遊戲和劇情解謎,感覺真的很好玩,我都想報名當小隊員。做出這些活動的人太厲害了。

\item \textbf{瘋狗} \\
隨緣過的生活常因科系間的隔閡而產生一堆麻煩。

\item \textbf{兩光博士} \\
科學知識和故事劇情是複雜的,但是喜歡科學的心卻可以是單純的唷!
\end{itemize}

\section{文宣組}
\begin{itemize}
\item \textbf{熊熊} \\
Hi大家,我是文書組的熊熊,希望大家會滿意這次的活動和課程,畢竟我們可是很努力的在達成這件事喔!

\item \textbf{無尾熊} \\
在這四天的營期裡,我們安排了很多有趣(?的課程,希望你們都可以學到東西,並且留下難忘的回憶,期待以後你們可以當我們的學弟妹喔$\sim$

\item \textbf{高速低能計算機} \\
很高興能夠在這個營隊和大家相遇,希望你們有獲得滿滿的科學知識及熱情。
\end{itemize}

\section{攝影組}
\begin{itemize}
\item \textbf{飛天章魚燒} \\
雪豹,真的,很可愛 \\
當然牠們也很漂亮(。・ω・。)ノ♡

不管怎樣希望各位看到照片的時候不會想打死業餘攝影組。 \\
還有感謝各位參加準備的老師同學工作人員,當然最重要的還是來參加的各位!沒有你們,我們做不到!

\item \textbf{廉價勞工}
很榮幸參加這次的活動並為團隊付出,希望下次有機會再和大家一起辦活動。
\end{itemize}

\section{隊輔}

\begin{itemize}
\item \textbf{菜䔕} \\
讓我們用手拉力去推廣科學$\sim\sim$

\item \textbf{肚毛} \\
大家辛苦了,希望在這4天裡面大家都有學到有用的知識,不論是學術的知識,或是身體的姿勢,希望未來大家都可以好好利用在這裡的所學,很開心可以當你們的領隊( • ̀ω•́ )✧

\item \textbf{羚羊} \\
總是在逃避中學習面對,在耍廢之中增進自我(??),一回神,一年就過了。腦海留下的是恐怖的公式,以及無數白癡又珍貴的回憶(包括這次科推)。

\item \textbf{大吉吉} \\
每一次的檢驗,都是為了使課程更生動,讓活動更有趣。\\
每一次的摩擦,都使我們更成熟 ,讓營隊更精彩。\\
我們成就科學推廣隊,科學推廣隊造就我們。\\
期待它也能帶給你們收穫。

\item \textbf{毛毛} \\
毛毛是個好孩子。
\item \textbf{販菸} \\
科推才能賦予我真正的定義 \\
ㄉㄨㄛㄩㄝㄇㄧ變身成音樂才女 \\
擺脫疑似瘧疾的緬甸絕症的束縛 \\
飛向沒有緬甸蒼蠅和鋼鐵韓粉的天空

\item \textbf{小飛象章魚} \\
開始準備科推的事,已經是去年的事了吧。在這期間,一次次的淬鍊,一次次的挫敗,只為最後的展翅高飛。願所有參與其中的學員們,都能夠得到當初在報名表單上填上的,希望從這次營隊中獲得的東西。除此之外,希望大家也能夠學到更多不同的東西,不僅限於課程的內容,而是從營隊生活的這幾天中的各種大大小小的事情,去觀察,去發現,去歸納整理出一條屬於自己的明天。

\item \textbf{大孢子葉癒合} \\
8月中的一個營隊——真的很熱,不過大家玩得開心就好。突然發現暑假快結束了,就把這當作暑假最後的回憶吧。

\item \textbf{菊花} \\
Crazy is putting it all one the line for the game you love. \\
我喜歡探險

\item \textbf{柳橙汁} \\
一年前的我看著學長姐們在暑假辦科推這個大活動,心裡真的很期待隔年我們自己辦的營隊,不知不覺中103一起走過一起成長一起在時間的長河中航過了一波一波的浪。從一群被提前抓來暑輔的陌生人,到園遊會前大家留下來在天黑了的教室研發鬆餅,創意歌唱比賽速產歌曲外加大鬧觀眾席,迎新水球戰的水花點點倒映著盛放的笑顏,這一年裡我們團結了許多,也懂事了許多,期待已久科推也悄悄的來臨,但是真的開始籌備的時候卻突然沒有什麼方向。當然了我的工作是隊輔比較沒有那麼吃重,很謝謝那些負責重要崗位的同學們,雖然我的生輔組不是很厲害的學科型,但是希望我負責的晚會大家可以玩的開心!!!大家那天晚上都要開開心心帥帥美美的不要醜醜的,全部都high起來喔!最後啊希望大家在我們這個營隊有學到科學方面的知識,也能有所成長。然後我要給親愛的3班一小段話,你們每個人都太棒了,準備出這些具深度的課程,當然了最重要的也是因為大家的同心協力才能完成這麼盛大的一個活動!!!

\item \textbf{朝朝} \\
郭
\item \textbf{綠帽} \\
科推歷經數個月的準備,在這期間我們不斷的討論、重複的練習,我們歷經失敗,爭吵,甚至一度以為要開天窗,但相信這一切會成為經驗,讓我們的科推變的更加成功。

\item \textbf{地獄棻妮} \\
全班一起策劃這次的活動,過程中有歡笑也有汗水(熱啊),受益良多啊!希望會很成功~
\end{itemize}

\section{事務執掌}
\begin{itemize}
\item \textbf{總召}:林語辰
\item \textbf{副召}:吳冠廷
\item \textbf{活動}:張智閎、吳冠廷、丁安磊、陸柏丞
\item \textbf{美宣}:李書妍、林語辰、洪嘉聲、李訓至
\item \textbf{庶務}:陳苙瑋、程品奕、張竣程、阮柏翰
\item \textbf{文書}:李杰穎、莊翔鈞、黃裕盛
\item \textbf{攝影}:高語儂、廖恩莆
\item \textbf{隊輔}:蔡博恩、賴城諭、林陽、胡睿喆、黃智笙、黃芃嫣、邱柏偉、葉欲禾、鄧駿樺、柳凱馨、鄧朝語、戴佑丞、張智閎、蘇郁棻
\end{itemize}

\section{教學執掌}
\begin{itemize}
\item \textbf{數學組}:張智閎、陸柏丞、黃裕盛、莊翔鈞、廖恩莆、葉欲禾
\item \textbf{物理組}:丁安磊、李杰穎、賴城諭、鄧朝語、戴佑丞、吳冠廷
\item \textbf{化學組}:張竣程、陳苙瑋、林陽、阮柏翰、胡睿喆、蘇郁棻
\item \textbf{生物組}:邱柏偉、高語儂、蔡博恩、李訓至、程品奕、洪嘉聲
\item \textbf{生輔組}:林語辰、黃芃嫣、黃智笙、柳凱馨、李書妍、鄧駿樺
\end{itemize}

\section{手冊製作}
\begin{itemize}
\item \textbf{編輯}:李杰穎
\item \textbf{封面及章節設計}:林語辰
\end{itemize}

